% 問題点
\expandafter\ifx\csname MAIN\endcsname\relax
  \documentclass[a4paper]{jarticle}
  \usepackage[top=2cm, bottom=2cm, left=2cm, right=2cm]{geometry}
  \newcommand{\resp}[1]{\begin{flushright}文責:~#1\end{flushright}~\\}
  \begin{document}
\fi

\section{問題点}
  \resp{原 佑馬}

  本プロジェクトにおける問題点として,次の3つの点が挙げられる.

  \begin{itemize}
    \item グループワークにおける時間の確保が困難であった.
    
      本プロジェクトでは,Ren'pyの機能を学習すると共に,グループワークを行い,ビジュアルノベルゲームの制作を行ったが,
      本格的に制作に取り掛かるまでに複数回活動を重ねる必要があり,結果的に制作に費やせる時間が短くなってしまった.
    
    \item 集団開発に関する知識を共有する機会がなかった.
    
      本プロジェクトにおけるグループ活動は,集団開発に関する見識を深める目的もあったが,これに使用するGitを用いたバージョン管理や,
      GitHubを用いた協働開発の方法に関する紹介は行わなかった.

      これは,先述の通りグループワークにおいて実際に制作を開始するまでにRen'pyについての紹介を終えておく必要があったために
      このような回を設けることができなかったためである.

      しかし,これらの技術について知識の無い班員が共同開発において遅れをとってしまうという事態も発生してしまった.
    
    \item 成果物の共有ができなかった.
    
      本プロジェクトでは,本来最終活動において,各グループの成果物を発表する場を設ける予定であった.

      しかし,最終的に各グループの成果物の完成が送れてしまったため,そのような場を設けることは叶わなかった.
  \end{itemize}

\expandafter\ifx\csname MAIN\endcsname\relax
  \end{document}
\fi
