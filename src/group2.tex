% 桃鶏肉ハンバーグ
\expandafter\ifx\csname MAIN \endcsname\relax
  \documentclass[a4paper]{jarticle}
  \usepackage[top=2cm, bottom=2cm, left=2cm, right=2cm]{geometry}
  \newcommand{\resp}[1]{\begin{flushright}文責:~#1\end{flushright}~\\}
  \begin{document}
\fi

\subsubsection{グループ2}
    
    本グループは,坪倉奏太をリーダーとして,岡本陽太,新藤尚輝のメンバーで構成された.

    \begin{itemize}
        \item グループ活動の進め方
        
        \resp{(文責者名)}

        % ここに文章

        \item 成果物概要
        
        \resp{(坪倉奏太)}

        BKCの各学部を擬人化し,情理を主人公とした恋愛シミュレーションゲーム
	攻略ヒロインは経済,スポ健,食マネ,生命・薬(一括り)
	一回生4月から,1週ごとを1ターンとし,16ターン終了後にエンディング
	各選択をどれだけ行ったかでエンディングが変わる

	なお,時間不足により実装は経済,食マネルートに留まった.これは,シナリオ制作の時間的見通	しが想定より甘かったことなどが原因として挙げられる.筆者は今回が初めてのシナリオを執筆し	た経験だったのだが,展開の論理的整合性を取るために多大な労力を必要とした.また,今回制作	したノベルゲームのジャンルが恋愛シミュレーションというより展開の論理的整合性が重要とされ	るものだったためさらに時間を要してしまった.


        \item 工夫点
        
        \resp{(文責者名)}

        % ここに文章

        \item 問題点
        
        \resp{(文責者名)}

        % ここに文章

        \item 展望
        
        \resp{(坪倉奏太)}

        今後の展望としては,まずは残りのルートの実装が挙げられる.また,ただ単にヒロインの画像と	テキストを表示するのではなく,カットインを導入する,ブラウザゲーム「艦隊これくしょん」で	実装されているような,キャラを小さく収縮,拡大することのよって各ヒロインがまるで呼吸をし	ているかのように見えるようにする,などと言った演出面の改善が挙げられる.

    \end{itemize}

\expandafter\ifx\csname MAIN \endcsname\relax
  \end{document}
\fi
