% 桃鶏肉ハンバーグ
\expandafter\ifx\csname MAIN \endcsname\relax
  \documentclass[a4paper]{jarticle}
  \usepackage[top=2cm, bottom=2cm, left=2cm, right=2cm]{geometry}
  \newcommand{\resp}[1]{\begin{flushright}文責:~#1\end{flushright}~\\}
  \begin{document}
\fi

\subsubsection{グループ2}
    
    本グループは,(リーダー名)をリーダーとして,(他のメンバー名)のメンバーで構成された.

    \begin{itemize}
        \item グループ活動の進め方
        
        \resp{(文責者名)}

        % ここに文章

        \item 成果物概要
        
        \resp{(文責者名)}

        % ここに文章

        \item 工夫点
        
        \resp{(岡本陽太)}

        % ここに文章
        王道な雰囲気で話を進めるためによりシンプルな設計にした.

        \item 問題点
        
        \resp{(岡本陽太)}

        % ここに文章
        renpy側とpython側とのつながりがよくわからなかったので雑な設計になった.またノベルゲームならではの絵も自作することができず,デザインに力をいれることができなかった.

        \item 展望
        
        \resp{(文責者名)}

        % ここに文章

    \end{itemize}

\expandafter\ifx\csname MAIN \endcsname\relax
  \end{document}
\fi
