% 桃鶏肉ハンバーグ
\expandafter\ifx\csname MAIN \endcsname\relax
  \documentclass[a4paper]{jarticle}
  \usepackage[top=2cm, bottom=2cm, left=2cm, right=2cm]{geometry}
  \newcommand{\resp}[1]{\begin{flushright}文責:~#1\end{flushright}~\\}
  \begin{document}
\fi

\subsubsection{グループ2}
    
    本グループは,(リーダー名)をリーダーとして,(他のメンバー名)のメンバーで構成された.

    \begin{itemize}
        \item グループ活動の進め方
        
        \resp{新藤 尚輝}

         私たちのグループは,それぞれの実現したいことを話あって,大まかな方針を固めた.その後,登場キャラやジャンルなど案の詳細を決めた.そして,プロットやストーリーの大まかな流れをまとめた.また,一度に全てを作ろうとするのではなく,ひとまず一つのルートを実装するという方針で進めていった.実装する際には,ストーリー担当とプログラム担当,イラスト(素材)担当にひとまず分かれて制作した.

        \item 成果物概要
        
        \resp{(文責者名)}

        % ここに文章

        \item 工夫点
        
        \resp{(文責者名)}

        % ここに文章

        \item 問題点
        
        \resp{(文責者名)}

        % ここに文章

        \item 展望
        
        \resp{(文責者名)}

        % ここに文章

    \end{itemize}

\expandafter\ifx\csname MAIN \endcsname\relax
  \end{document}
\fi
