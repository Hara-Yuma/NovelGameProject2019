\subsubsection{グループ2}

    本グループは,坪倉奏太をリーダーとして,岡本陽太,新藤尚輝のメンバーで構成された.

        \subsubsubsection{グループ活動の進め方}
        \resp{新藤 尚輝}

        本グループでは意見交換を行うことで,徐々に作品の方向性を固め,その後,1つのルートに絞って実装を行った.
        
        実装の際には,ストーリー担当とプログラム担当,イラスト(素材)担当に分かれて制作した.

        \subsubsubsection{成果物概要}
        \resp{坪倉 奏太}

        本学びわこ・くさつキャンパスの各学部を擬人化し,情報理工学部学生を主人公とした恋愛シミュレーションゲームである.

        攻略対象ヒロインは経済,スポ健,食マネ,生命・薬(一括り)である.
        
        1回生4月から,1週ごとを1ターンとし,16ターン終了後にエンディング各選択をどれだけ行ったかでエンディングが変わる.

        なお,時間不足により実装は経済,食マネルートに留まった.
      
        \subsubsubsection{工夫点}
        \resp{岡本 陽太}

        王道な雰囲気で話を進めるためによりシンプルな設計にした.

        \subsubsubsection{問題点}
        \resp{岡本 陽太}

        Ren'PyにおけるPythonの変数などの扱いが複雑であり,そのために少々設計が荒くなってしまった部分があった.

        また,当初の予定に反して,ゲーム内で使用する絵などの素材を自作するなど,デザインに力をいれることができなかった.

        \subsubsubsection{展望}
        \resp{坪倉 奏太}

        今後の展望として,実装が間に合わなかったルートの実装や,立ち絵に動きを付けたり,カットインを導入するなどといった演出面の改善を行うことが挙げられる.
