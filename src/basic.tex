% Ren'pyの基本的な使い方
\expandafter\ifx\csname MAIN\endcsname\relax
  \documentclass[a4paper]{jarticle}
  \usepackage[top=2cm, bottom=2cm, left=2cm, right=2cm]{geometry}
  \newcommand{\resp}[1]{\begin{flushright}文責:~#1\end{flushright}~\\}
  \begin{document}
\fi

\subsection{Ren'pyの基本的な使い方}
  \resp{小柳 雅文,林 紘也}

  Ren'Pyの基本的な機能としては,次のものが挙げられる.

  \begin{itemize}
    \item ラベルと制御フロー
    \item 台詞とナレーション
    \item 画像の表示
    \item 選択肢
    \item オーディオ
  \end{itemize}

  本項目では,これらについて順に説明していく.

  \subsubsection{ラベルと制御フロー}

    Ren'pyのスクリプトでは, labelが1つの区切りとなる.
    
    labelは,スクリプト内の任意の場所につけることができ,これらはjump命令などを利用して移動することができる.
    ゲーム内における分岐などはこれを用いて行う.

  \subsubsection{台詞とナレーション}

    まずは,Ren'pyスクリプトにおけるキャラクターの定義について説明する.

    Ren'pyのスクリプト内では,キャラクターの表示名と表示色を定義することができ,
    これを用いることで,簡単にキャラクターに台詞を与えることが可能となる.

    例えば,表示名がキャラクターで,表示色が\#ffffffであるキャラクターを'c'として定義する場合,

    ~

    define c = Character("キャラクター", color="\#ffffff")

    ~

    のようにして行う.
    また,Ren'pyのスクリプト内で,

    ~

    "ナレーション"

    ~

    のようにダブルクォーテーションマークで囲われた文字列を記述した場合,これはナレーション
    (特別話し手が存在しない台詞)として表示される.
    更に,定義したキャラクター名を用いて,

    ~

    c "キャラクターの台詞"

    ~

    のように記述することで,任意のキャラクターに対して台詞を与えることが可能である.

  \subsubsection{画像の表示}

    Ren'pyでは,背景やキャラクターの画像を表示することも非常に容易である.
    例えば,scene命令では背景を表示することができ,show命令ではキャラクターを表示することができる.

    Ren'pyにおいて,画像は全て予め生成されるimagesフォルダ内に設置し,タグと属性を半角スペース,もしくはアンダーバーで区切って命名する.
    例えば,"bg background"という画像のタグは"bg"であり,属性は"background"となる.
    このようにタグと属性に分けることで,画像の削除(hide)などにおいて,タグを指定するだけで操作を行うことができるという利点がある.

    また,atやwithといったステートメントと併用することで,画像の表示位置や,効果を細かく指定することができる.

  \subsubsection{選択肢}

    menuステートメントを用いることで,選択肢を設けることができる.
    
    menuステートメントでは,表示する項目名と,その項目が選択された際に実行するブロックを設定することができる.

  \subsubsection{オーディオ}

    BGMやSE再生も,play・stopなどの命令を用いて簡単に行うこともできる.

    また,キャラクターに対してボイスを設定することもできるため,本格的なボイス付きノベルゲームを作成することも可能である.

  \subsubsection{GUI}

    プロジェクト内のgui.rpyファイルや,用意されている画像を編集・置換することで,ゲーム画面の大きさや文字の色,大きさなど
    数多くの項目について調整することができる.

  ~

  このようにRen'Pyにはノベルゲームを作るうえで必要な機能が整っている.

\expandafter\ifx\csname MAIN \endcsname\relax
  \end{document}
\fi
