% Ren'pyの拡張性
\expandafter\ifx\csname MAIN\endcsname\relax
  \documentclass[a4paper]{jarticle}
  \usepackage[top=2cm, bottom=2cm, left=2cm, right=2cm]{geometry}
  \newcommand{\resp}[1]{\begin{flushright}文責:~#1\end{flushright}~\\}
  \begin{document}
\fi

\section{Ren'pyの拡張性}
  \resp{新藤 尚輝}

  ここでは,Ren'Pyの拡張性について次の順に触れていく.
  
  \begini{itemize}
    \item  ステートメントの等価性
    \item 台詞
  \item  選択肢
    \item  画像の表示
    \item  トランジション
    \item  Jump
    \item  Call

  \end{itemize}

  \subsubsection{ステートメントの等価性}
   Ren'PyではPythonでスクリプト化できるように,各Ren'PyステートメントがPythonコードで記述できるようになっている.これは,通常,Python関数で構成されているが,ステートメントと同等のものを実行するPythonのパターンによることもある.
  \subsubsection{台詞}
   sayステートメントと同じものとして,次のように記述することができる.

   ~
   renpy.say(who,what,*args,**kwargs)
   ~

   who:発言するキャラクター,または,キャラクター名の文字列.Noneの場合ナレーターになる.
   what:発言する文字列.
   interact:この値が真の場合,ダイアログを表示するときにプレイヤーの入力を待つ.

   以下のコードは,それぞれ同様の動作になる.

   ~
   e "Hello, world."
   $ renpy.say(e, "Hello, world.")
   $ e("Hello, world.")
   ~

  \subsubsection{選択肢}
   menuステートメントにも等価なPython関数がある.
  
   ~
   menu(items,interact=True,screen="choice")
   renpy.display_menu(items, interact=True, screen="choice")
   ~
  
   items:ここには2つの要素のタプルのリストが入る.各タプルの最初の要素はラベル.2つ目の要素は選択されたときに返えされる値.値がNoneの時は,最初の要素はキャプションに使用される.
   interact:Falseにすると,選択肢を表示するが,インタラクションが実行されない.
   screen:選択肢を表示するために使用されるスクリーンの名前.

  \subsubsection{画像の表示}
   image,scene,showとhideステートメントには,それぞれ等価なPython関数が存在している.

    ~
    renpy.get_at_list(name, layer=None)   
    ~
    レイヤー上のnameタグの画像に適用されている変換の地ストを返す.変換されていないときは空のリストを返し,画像が表示されていない時はNoneを返す.
   レイヤーがNoneの時は,指定されたタグのデフォルトレイヤーを使用する.

   hideについて:
   ~
renpy.hide(name, layer=None)
   
   ~
   レイヤーから画像を非表示にする.hideステートメントと等価なPython関数.

    name:非表示にする画像の名前.画像タグのみを使用し,そのタグのすべての画像が非表示になる.
    layer:この関数が処理するレイヤー.Noneの時は,タグのデフォルトレイヤーを使用.

    imageについて:
    ~
    renpy.image(name, d)
    ~
    画像を定義する.imageステートメントと等価なPython関数.この関数は初期化(init)ブロック内でのみ実行される.

    name:表示する画像の名前.文字列を入力する.
    d:その画像名に関連付けられるdisplayable.

    sceneについて:
    ~
    renpy.scene(layer='master')
    ~
    レイヤーからすべてのdisplayableを除去する.表示する画像が与えられない時のsceneステートメントと等価.

    sceneステートメントは
    ~
    scene bg beach
    ~
    のように表示するが,これは
    ~
    $ renpy.scene()
    $ renpy.show("bg beach")
    ~
    と等価になる.

    showについて:
    ~
    renpy.show(name, at_list=, []layer='master', what=None, zorder=0, tag=None, behind=[])
    ~
    レイヤーに画像を表示する.showステイトメントと等価.

    name:表示する画像の名前.文字列を入力する.
    at_list:画像に適用される変換のリスト.atプロパティと等価.
    what:Noneでないならば,画像を探す代わりに表示されるdisplayableになる(show expressionステイトメントと等価).この引数が与えられると,nameは,その画像とタグを関連付けるために使用される.
    zorder:zorderプロパティと等価な整数.Noneの時は,zoderがあれば維持され,そうでなければ0に設定される.
    tag:表示される画像の画像タグを指定するために使用される文字列.asプロパティと等価.
    behind:背後に表示される画像タグの文字列のリスト.behindプロパティと等価.

    ~
    renpy.show_layer_at(at_list, layer='master', reset=True) 
    ~
    show layer layer(上記コードのlayer) at at_list(上記コードのat_list)ステートメントのPython関数での書き方.

    reset:真の時は,変換状態が表示された時に,始めの状態にリセットされる.

  \subsubsection{トランジション}
   ~
   renpy.with_statement(trans, always=False)
   ~
   トランジションを実行する.withステートメントと等価なPython関数.
   この関数はユーザーがトランジションの中断を選択すると,Trueを返す.

   trans:トランジション
   always:Trueなら,ユーザーがトランジションを無効化していても常に実行する.

  \subsubsection{Jump}
    jumpステートメントはrenpy.jump関数になる.
  ~
    renpy.jump(label)
    ~
    現在のステートメントを終了し,指定されたラベルに遷移する.
    
  \subsubsection{Call}
    コールステートメントはrenpy.call関数になる.

    ~
    renpy.call(label, *args, **kwargs) 
    ~
    現在のRen'Pyステートメントを終了し,指定されたラベルにジャンプする.ジャンプから戻ると,現在のステートメントに続くステートメントに制御が移る.

    from_current:真の場合,制御は現在のステートメントに続くステートメントではなく現在のステートメントに戻る(現在のステートメントが二回実行される).キーワード引数として渡す必要がある.

    ~
    renpy.return_statement() link
    ~
    現在のRen'Pyの呼び出しレベルからRen'Pyを返す.



\expandafter\ifx\csname MAIN \endcsname\relax
  \end{document}
\fi
