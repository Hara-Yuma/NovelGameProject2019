\subsubsection{グループ0}
    
    本グループは,原 佑馬をリーダーとして,中川 拓海,林 紘也,桐井 優実の4人メンバーで構成された.

    \begin{itemize}
        \item グループ活動の進め方
        
        \resp{原 佑馬}

        本グループでは,メインシナリオと素材の選定担当1人,サブシナリオとその実装担当2人,ゲームのベースとメインシナリオの実装担当1人に
        担当を分担し,制作活動を行った.

        制作物における登場人物や,ストーリーについては予めある程度の認識共有を行い,ストーリー間にできるだけ矛盾が生じたりしないように努めた.

        毎週1回設けたグループ活動では,基本的に本グループの活動ではある程度の到達目標を設定し,各々が作業してGitHubを利用して1つの作品を作り上げるという形をとった.

        \item 成果物概要
        
        \resp{中川 拓海}

        % ここに文章

        \item 工夫点
        
        \resp{原 佑馬}

        本グループの成果物では,メインシナリオとサブシナリオに分割し,シナリオ作成と実装を行った.

        また,ゲームのシステムとして1度で全てのシナリオをプレイできないようにし,更にそこに登場するキャラクターを選択できるようにすることで,
        何周かプレイしてもらえるように工夫している.

        \item 問題点
        
        \resp{原 佑馬}

        本グループでは,非常に膨大な量のシナリオを実装したため,まだまだ修正すべき点や,工夫できる点が存在していると考えられる.

        また,メインとサブのシナリオに分割し,各々が同じタイミングで作成してしまったことによって,多少なりともキャラクターの
        発言や行動に違和感が生じる結果になってしまっている.

        \item 展望
        
        \resp{中川 拓海}

        % ここに文章

    \end{itemize}

