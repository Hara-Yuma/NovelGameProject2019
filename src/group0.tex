\subsubsection{グループ0}
    
    本グループは,原 佑馬をリーダーとして,中川 拓海,林 紘也,桐井 優実の4人メンバーで構成された.

    \subsubsubsection{グループ活動の進め方}
    \resp{原 佑馬}
    本グループでは,メインシナリオと素材の選定担当1人,サブシナリオとその実装担当2人,ゲームのベースとメインシナリオの実装担当1人に
    担当を分担し,制作活動を行った.

    制作物における登場人物や,ストーリーについては予めある程度の認識共有を行い,ストーリー間にできるだけ矛盾が生じたりしないように努めた.

    毎週1回設けたグループ活動では,基本的に本グループの活動ではある程度の到達目標を設定し,各々が作業してGitHubを利用して1つの作品を作り上げるという形をとった.

    ~

    \subsubsubsection{成果物概要}
    \resp{原 佑馬}
    本グループの成果物は,プレイヤーの選択によってヒロインやその好感度が変化するようなビジュアルノベルゲームとなった.

    本来であればメインストーリー1章に対してサブストーリーが2つあり,その内のどちらかをプレイヤーが選択してプレイすることができるシステムとする予定であったが,
    時間の都合上実装することはできなかった.

    また,ヒロインは複数人おり,それぞれが対応するストーリーをクリアすることで開放される.
    
    フラグを用いた分岐や,時間計測,入力機能を用いた探索パートや,謎解き要素なども盛り込んでいる.

    \subsubsubsection{工夫点}
    \resp{原 佑馬}

    本グループの成果物では,メインシナリオとサブシナリオに分割し,シナリオ作成と実装を行った.

    また,ゲームのシステムとして1度で全てのシナリオをプレイできないようにし,更にそこに登場するキャラクターを選択できるようにすることで,
    何周かプレイしてもらえるように工夫している.

    \subsubsubsection{問題点}
    \resp{原 佑馬}

    本グループでは,非常に膨大な量のシナリオを実装したため,まだまだ修正すべき点や,工夫できる点が存在していると考えられる.

    また,メインとサブのシナリオに分割し,各々が同じタイミングで作成してしまったことによって,多少なりともキャラクターの
    発言や行動に違和感が生じる結果になってしまっている.

    \subsubsubsection{展望}
    \resp{原 佑馬}

    展望としては,実装できなかったシナリオの実装などが挙げられる.

    そのため,本グループのプロジェクトは2020年度の学園祭において展示することを目標に引き続き開発を行うこととする.
