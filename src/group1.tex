\subsubsection{グループ1}
    
    本グループは,中山凌一をリーダーとして,小柳雅文,佐藤祐樹のメンバーで構成された.

    \begin{itemize}
        \item グループ活動の進め方
        
        \resp{中山凌一}

        本グループは,担当を「シナリオ制作」「コーディング」「デザイン編集」に分け,制作を行った.

        「シナリオ制作」は,初週に班員で決め,作品の舞台設定やキャラ設定などをもとにメインストーリーを制作し,
        分岐点やサブストーリーを考案した.

        「コーディング」は,シナリオをゲームとして動くようにコーディングを行った.

        「デザイン編集」は,キャラに合わせた立ち絵や背景画像の準備,ゲーム全体の見た目を考えた.

        \item 成果物概要
        
        \resp{中山凌一}

        本ゲームは,大学を舞台とし,そこで出会う5人の女性と日々を過ごしながら,主人公の行動によって変化する
        彼女たちの好感度を高め,エンディングを迎えるようになっている.

        しかし,現段階では制作時間の関係で分岐ストーリーは無く,一人の女性と好感度を高め,エンディングを迎える仕様になっている.
        
        ゲームは,定期的に出現する,主人公の行動を決定する選択肢を選んで行くことで進行する.この選択肢によって女性の好感度が変化し,
        好感度にあったエンディングを迎えるようになっている.

        \item 工夫点
        
        \resp{中山凌一}

        ゲームの最初に主人公の名前を入力する場所があり,そこで入力した名前がゲーム内に反映されるようになっている.

        \item 問題点
        
        \resp{中山凌一}

        時間の関係上,使用した素材の利用規約等を全て確認できていない.

        そのため,本制作物を公開する際には利用規約を確認し,問題があればその箇所について変更する必要がある.

        \item 展望
        
        \resp{中山凌一}

        今回,本制作物では,分岐等が存在せず,少し単調なものになってしまっている.

        そのため,選択肢を用いた分岐等を実装することなどが考えられる.

    \end{itemize}
