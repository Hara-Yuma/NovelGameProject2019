\subsubsection{グループ1}
    
    本グループは,中山凌一をリーダーとして,小柳雅文,佐藤祐樹のメンバーで構成された.

    \subsubsubsection{グループ活動の進め方}
    \resp{中山 凌一}

    本グループは,担当を「シナリオ制作」「コーディング」「デザイン編集」に分け,制作を行った.

    「シナリオ制作」は初週に班員で決めた作品の舞台設定やキャラ設定などをもとにメインストーリーを制作し,
    分岐点やサブストーリーを考案した.

    「コーディング」はシナリオをゲームとして動くようにコーディングを行った.

    「デザイン編集」はキャラに合わせた立ち絵や背景画像の準備,ゲーム全体の見た目を考えた.

    \newpage

    \subsubsubsection{成果物概要}
    \resp{中山 凌一}

    本ゲームは大学を舞台とし,そこで出会う5人の女性と日々を過ごしながら主人公の行動によって変化する
    彼女たちの好感度を高めてエンディングを迎えるようになっている.

    しかし,現段階では制作時間の関係で分岐ストーリーは無く,一人の女性と好感度を高めてエンディングを迎える仕様になっている.
    
    本ゲームは,定期的に出現する主人公の行動を決定する選択肢を選んで行くことで進行する.

    この選択肢によって女性の好感度が変化し,好感度にあったエンディングを迎えるようになっている.

    \subsubsubsection{工夫点}
    \resp{中山 凌一}

    ゲームの最初に主人公の名前を入力する場所があり,そこで入力した名前がゲーム内に反映されるようになっている.

    \subsubsubsection{問題点}
    \resp{中山 凌一}

    時間の関係上,使用した素材の利用規約等を全て確認できていない.

    そのため,本制作物を公開する際には利用規約を確認し,問題があればその箇所について変更する必要がある.

    \subsubsubsection{展望}
    \resp{中山 凌一}

    今回,本制作物では,分岐等が存在せず,少し単調なものになってしまっている.

    そのため,選択肢を用いた分岐等を実装することなどが考えられる.
