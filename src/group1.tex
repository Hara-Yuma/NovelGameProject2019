% rks
\expandafter\ifx\csname MAIN \endcsname\relax
  \documentclass[a4paper]{jarticle}
  \usepackage[top=2cm, bottom=2cm, left=2cm, right=2cm]{geometry}
  \newcommand{\resp}[1]{\begin{flushright}文責:~#1\end{flushright}~\\}
  \begin{document}
\fi

\subsubsection{rks}
    
    本グループは,中山凌一をリーダーとして,小柳雅文,佐藤祐樹のメンバーで構成された.

    \begin{itemize}
        \item グループ活動の進め方
        
        \resp{中山凌一}

        本グループは,担当を「シナリオ制作」「コーディング」「デザイン編集」に分け,制作を行った。
        「シナリオ制作」は,初週に班員で決めた,作品の舞台設定やキャラ設定などをもとにメインストーリーを制作し,
        分岐点やサブストーリーを考えた。
        「コーディング」は,シナリオをゲームとして動くようにコーディングを行った。
        「デザイン編集」は,キャラに合わせた立ち絵や背景画像の準備,ゲーム全体の見た目を考えた。

        \item 成果物概要
        
        \resp{中山凌一}

        本ゲームは,大学を舞台とし,そこで出会う5人の女性と日々を過ごしながら,主人公の行動によって変化する
        彼女たちの好感度を高め,エンディングを迎えるようになっている。
        しかし,現段階では制作時間の関係で分岐ストーリーは無く,一人の女性と好感度を高め,エンディングを迎える仕様になっている。
        ゲームは,定期的に出現する,主人公の行動を決定する選択肢を選んで行くことで進行する。この選択肢によって女性の好感度が変化し,
        好感度にあったエンディングを迎えるようになっている。

        \item 工夫点
        
        \resp{中山凌一}

        ゲームの最初に主人公の名前を入力する場所があり,そこで入力した名前がゲーム内に反映されるようになっている。

        \item 問題点
        
        \resp{中山凌一}

        時間の関係上,使用した背景及びbgmの一部において利用規約を読まずに使っている。
        この点について,利用規約を確認し,問題があればその箇所について変更する必要がある。

        \item 展望
        
        \resp{中山凌一}

        今回,複数人の女性と過ごすはずの大学生活が一人の女性と過ごす事しかできずに終わってしまっている。
        今後,他の女性のストーリーの追加やエンディングの追加を行ってきたいと思う。

    \end{itemize}

\expandafter\ifx\csname MAIN \endcsname\relax
  \end{document}
\fi
