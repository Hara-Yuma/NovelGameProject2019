% Ren'pyについて
\expandafter\ifx\csname MAIN\endcsname\relax
  \documentclass[a4paper]{jarticle}
  \usepackage[top=2cm, bottom=2cm, left=2cm, right=2cm]{geometry}
  \newcommand{\resp}[1]{\begin{flushright}文責:~#1\end{flushright}~\\}
  \begin{document}
\fi

\subsection{Ren'pyについて}
  \resp{佐藤 祐樹}

  Ren'Pyとは,ヴィジュアルノベルゲームを開発するためのゲームエンジンである.
2004年8月24日にリリースされ,その後も更新が続いている.
オープンソースソフトウェアであり,商用・非商用問わず無料で利用することが可能である.
また,Ren'Pyは,簡単なスクリプト言語を用いて開発することが可能である.
Ren'Pyには,既読テキストスキップ機能,自動セーブ機能等の,一般にヴィジュアルノベルゲーム開発で用いられる機能がデフォルトで含まれている.
これらの機能により,容易にヴィジュアルノベルゲームを開発することが可能である.
また,スクリプト中にPythonコードを記述することにより,上述したような機能のカスタマイズ,Ren'Pyに含まれていない画面効果の実装等が可能である.
したがって,Ren'Pyは手軽にヴィジュアルノベルゲームの開発が可能でありながら,高いカスタマイズ性を備えているといえる.

\expandafter\ifx\csname MAIN \endcsname\relax
  \end{document}
\fi
