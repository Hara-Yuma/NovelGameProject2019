\documentclass[a4paper]{jarticle}

\usepackage[top=2cm, bottom=2cm, left=2cm, right=2cm]{geometry}

\renewcommand\thefootnote{*\arabic{footnote}}

\newcommand{\resp}[1]{\begin{flushright}文責:~#1\end{flushright}~\\}
\newcommand{\MAIN}{true}

\title{ビジュアルノベル制作班 活動報告書}
\author{立命館コンピュータクラブ\\2019年度後期プロジェクト活動}
\date{2020年2月10日}

\begin{document}
  \maketitle

  \begin{center}
    原 佑馬 \footnote{情報理工学部 情報理工学科 SAコース 2回生}
  \end{center}

  \newpage
  \tableofcontents
  \newpage

  % 活動概要
\expandafter\ifx\csname MAIN\endcsname\relax
  \documentclass[a4paper]{jarticle}
  \usepackage[top=2cm, bottom=2cm, left=2cm, right=2cm]{geometry}
  \newcommand{\resp}[1]{\begin{flushright}文責:~#1\end{flushright}~\\}
  \begin{document}
\fi

\section{活動概要}
  \resp{原 佑馬}

  本プロジェクトでは,ビジュアルノベル制作エンジンであるRen'pyを使用してビジュアルノベルゲームを制作するための手法や,
  その際に必要となる著作権についての知識を付け,ネット上などで第三者に配布することができるようなゲームを作成するための知識及び技術を
  会得することを目標とし,週に2回の活動を行った.

  また,実際に得た知識を利用し,また,集団開発に関する見識を深めるため,プロジェクト内で3〜4人で構成されるグループを3つ作成し,
  各グループ毎に1つのビジュアルノベルゲームの開発を行った.

\expandafter\ifx\csname MAIN\endcsname\relax
  \end{document}
\fi

  
  \section{活動内容}

    本プロジェクト活動では,Ren'pyについての説明を行う全体活動と,グループに別れて1つの作品を制作するグループ活動を行った.
      
    本章では,それぞれの活動について,各項目に分けて報告する.

    % Ren'pyについて
\expandafter\ifx\csname MAIN\endcsname\relax
  \documentclass[a4paper]{jarticle}
  \usepackage[top=2cm, bottom=2cm, left=2cm, right=2cm]{geometry}
  \newcommand{\resp}[1]{\begin{flushright}文責:~#1\end{flushright}~\\}
  \begin{document}
\fi

\subsection{Ren'pyについて}
  \resp{佐藤 祐樹}

  Ren'Pyとは,ビジュアルノベルゲームを開発するためのゲームエンジンである.
2004年8月24日にリリースされ,その後も更新が続いている.
オープンソースソフトウェアであり,商用・非商用問わず無料で利用することが可能である.
また,Ren'Pyは,簡単なスクリプト言語を用いて開発することが可能である.
Ren'Pyには,既読テキストスキップ機能,自動セーブ機能等の,一般にビジュアルノベルゲーム開発で用いられる機能がデフォルトで含まれている.
これらの機能により,容易にビジュアルノベルゲームを開発することが可能である.
また,スクリプト中にPythonコードを記述することにより,上述したような機能のカスタマイズ,Ren'Pyに含まれていない画面効果の実装等が可能である.
したがって,Ren'Pyは手軽にビジュアルノベルゲームの開発が可能でありながら,高いカスタマイズ性を備えているといえる.

\expandafter\ifx\csname MAIN \endcsname\relax
  \end{document}
\fi


    % Ren'pyの基本的な使い方
\expandafter\ifx\csname MAIN\endcsname\relax
  \documentclass[a4paper]{jarticle}
  \usepackage[top=2cm, bottom=2cm, left=2cm, right=2cm]{geometry}
  \newcommand{\resp}[1]{\begin{flushright}文責:~#1\end{flushright}~\\}
  \begin{document}
\fi

\subsection{Ren'pyの基本的な使い方}
  \resp{小柳 雅文,林 紘也}

  Ren'Pyの基本的な機能としては,次のものが挙げられる.

  \begin{itemize}
    \item ラベルと制御フロー
    \item 台詞とナレーション
    \item 画像の表示
    \item 選択肢
    \item オーディオ
  \end{itemize}

  本項目では,これらについて順に説明していく.

  \subsubsection{ラベルと制御フロー}

    Ren'pyのスクリプトでは, labelが1つの区切りとなる.
    
    labelは,スクリプト内の任意の場所につけることができ,これらはjump命令などを利用して移動することができる.
    ゲーム内における分岐などはこれを用いて行う.

  \subsubsection{台詞とナレーション}

    まずは,Ren'pyスクリプトにおけるキャラクターの定義について説明する.

    Ren'pyのスクリプト内では,キャラクターの表示名と表示色を定義することができ,
    これを用いることで,簡単にキャラクターに台詞を与えることが可能となる.

    例えば,表示名がキャラクターで,表示色が\#ffffffであるキャラクターを'c'として定義する場合,

    ~

    define c = Character("キャラクター", color="\#ffffff")

    ~

    のようにして行う.
    また,Ren'pyのスクリプト内で,

    ~

    "ナレーション"

    ~

    のようにダブルクォーテーションマークで囲われた文字列を記述した場合,これはナレーション
    (特別話し手が存在しない台詞)として表示される.
    更に,定義したキャラクター名を用いて,

    ~

    c "キャラクターの台詞"

    ~

    のように記述することで,任意のキャラクターに対して台詞を与えることが可能である.

  \subsubsection{画像の表示}

    Ren'pyでは,背景やキャラクターの画像を表示することも非常に容易である.
    例えば,scene命令では背景を表示することができ,show命令ではキャラクターを表示することができる.

    Ren'pyにおいて,画像は全て予め生成されるimagesフォルダ内に設置し,タグと属性を半角スペース,もしくはアンダーバーで区切って命名する.
    例えば,"bg background"という画像のタグは"bg"であり,属性は"background"となる.
    このようにタグと属性に分けることで,画像の削除(hide)などにおいて,タグを指定するだけで操作を行うことができるという利点がある.

    また,atやwithといったステートメントと併用することで,画像の表示位置や,効果を細かく指定することができる.

  \subsubsection{選択肢}

    menuステートメントを用いることで,選択肢を設けることができる.
    
    menuステートメントでは,表示する項目名と,その項目が選択された際に実行するブロックを設定することができる.

  \subsubsection{オーディオ}

    BGMやSE再生も,play・stopなどの命令を用いて簡単に行うこともできる.

    また,キャラクターに対してボイスを設定することもできるため,本格的なボイス付きノベルゲームを作成することも可能である.

  \subsubsection{GUI}

    プロジェクト内のgui.rpyファイルや,用意されている画像を編集・置換することで,ゲーム画面の大きさや文字の色,大きさなど
    数多くの項目について調整することができる.

  ~

  このようにRen'Pyにはノベルゲームを作るうえで必要な機能が整っている.

\expandafter\ifx\csname MAIN \endcsname\relax
  \end{document}
\fi


    \section{Ren'pyの拡張性}
  \resp{原 佑馬}

  Ren'pyスクリプトでは,"\$"で開始される行及び"python:"で指定されたブロックがPythonスクリプトとして扱われる.

  Ren'pyスクリプト内で使用されるPythonスクリプトでは,変数が共有され,また,ゲームデータのセーブと共に,これら変数の変更が記録される.

  更に,Ren'pyでゲームの動作を定義するためのステートメントは,Pythonのコードによって記述されているため,
  Ren'pyスクリプトに埋め込んだPythonコード内でもキャラクターの表示やラベル間遷移,発話等を行うことができる.

  これにより,ユーザからの入力や,時刻,各フラグの値などによって発話するキャラクターや,
  その内容を変化させるなど幅広い動作を実現することが可能となっている.
  

    \subsection{各グループ活動報告}

      % 木曜どうでしょう
\expandafter\ifx\csname MAIN \endcsname\relax
  \documentclass[a4paper]{jarticle}
  \usepackage[top=2cm, bottom=2cm, left=2cm, right=2cm]{geometry}
  \newcommand{\resp}[1]{\begin{flushright}文責:~#1\end{flushright}~\\}
  \begin{document}
\fi

\subsubsection{グループ0}
    
    本グループは,(リーダー名)をリーダーとして,(他のメンバー名)のメンバーで構成された.

    \begin{itemize}
        \item グループ活動の進め方
        
        \resp{(文責者名)}

        % ここに文章

        \item 成果物概要
        
        \resp{(文責者名)}

        % ここに文章

        \item 工夫点
        
        \resp{(文責者名)}

        % ここに文章

        \item 問題点
        
        \resp{(文責者名)}

        % ここに文章

        \item 展望
        
        \resp{(文責者名)}

        % ここに文章

    \end{itemize}

\expandafter\ifx\csname MAIN \endcsname\relax
  \end{document}
\fi


      % rks
\expandafter\ifx\csname MAIN \endcsname\relax
  \documentclass[a4paper]{jarticle}
  \usepackage[top=2cm, bottom=2cm, left=2cm, right=2cm]{geometry}
  \newcommand{\resp}[1]{\begin{flushright}文責:~#1\end{flushright}~\\}
  \begin{document}
\fi

\subsubsection{rks}
    
    本グループは,中山凌一をリーダーとして,小柳雅文,佐藤祐樹のメンバーで構成された.

    \begin{itemize}
        \item グループ活動の進め方
        
        \resp{中山凌一}

        本グループは,担当を「シナリオ制作」「コーディング」「デザイン編集」に分け,制作を行った。
        「シナリオ制作」は,初週に班員で決めた,作品の舞台設定やキャラ設定などをもとにメインストーリーを制作し,
        分岐点やサブストーリーを考えた。
        「コーディング」は,シナリオをゲームとして動くようにコーディングを行った。
        「デザイン編集」は,キャラに合わせた立ち絵や背景画像の準備,ゲーム全体の見た目を考えた。

        \item 成果物概要
        
        \resp{中山凌一}

        本ゲームは,大学を舞台とし,そこで出会う5人の女性と日々を過ごしながら,主人公の行動によって変化する
        彼女たちの好感度を高め,エンディングを迎えるようになっている。
        しかし,現段階では制作時間の関係で分岐ストーリーは無く,一人の女性と好感度を高め,エンディングを迎える仕様になっている。
        ゲームは,定期的に出現する,主人公の行動を決定する選択肢を選んで行くことで進行する。この選択肢によって女性の好感度が変化し,
        好感度にあったエンディングを迎えるようになっている。

        \item 工夫点
        
        \resp{中山凌一}

        ゲームの最初に主人公の名前を入力する場所があり,そこで入力した名前がゲーム内に反映されるようになっている。

        \item 問題点
        
        \resp{中山凌一}

        時間の関係上,使用した背景及びbgmの一部において利用規約を読まずに使っている。
        この点について,利用規約を確認し,問題があればその箇所について変更する必要がある。

        \item 展望
        
        \resp{中山凌一}

        % ここに文章

    \end{itemize}

\expandafter\ifx\csname MAIN \endcsname\relax
  \end{document}
\fi


      % 桃鶏肉ハンバーグ
\expandafter\ifx\csname MAIN \endcsname\relax
  \documentclass[a4paper]{jarticle}
  \usepackage[top=2cm, bottom=2cm, left=2cm, right=2cm]{geometry}
  \newcommand{\resp}[1]{\begin{flushright}文責:~#1\end{flushright}~\\}
  \begin{document}
\fi

\subsubsection{グループ2}
    
    本グループは,坪倉奏太をリーダーとして,岡本陽太,新藤尚輝のメンバーで構成された.

    \begin{itemize}
        \item グループ活動の進め方
        
        \resp{(文責者名)}

        % ここに文章

        \item 成果物概要
        
        \resp{(坪倉奏太)}

        BKCの各学部を擬人化し,情理を主人公とした恋愛シミュレーションゲーム
	攻略ヒロインは経済,スポ健,食マネ,生命・薬(一括り)
	一回生4月から,1週ごとを1ターンとし,16ターン終了後にエンディング
	各選択をどれだけ行ったかでエンディングが変わる

	なお,時間不足により実装は経済,食マネルートに留まった.これは,シナリオ制作の時間的見通	しが想定より甘かったことなどが原因として挙げられる.筆者は今回が初めてのシナリオを執筆し	た経験だったのだが,展開の論理的整合性を取るために多大な労力を必要とした.また,今回制作	したノベルゲームのジャンルが恋愛シミュレーションというより展開の論理的整合性が重要とされ	るものだったためさらに時間を要してしまった.


        \item 工夫点
        
        \resp{(岡本陽太)}

        % ここに文章
        王道な雰囲気で話を進めるためによりシンプルな設計にした.

        \item 問題点
        
        \resp{(岡本陽太)}

        % ここに文章
        renpy側とpython側とのつながりがよくわからなかったので雑な設計になった.またノベルゲームならではの絵も自作することができず,デザインに力をいれることができなかった.

        \item 展望
        
        \resp{(坪倉奏太)}

        今後の展望としては,まずは残りのルートの実装が挙げられる.また,ただ単にヒロインの画像と	テキストを表示するのではなく,カットインを導入する,ブラウザゲーム「艦隊これくしょん」で	実装されているような,キャラを小さく収縮,拡大することのよって各ヒロインがまるで呼吸をし	ているかのように見えるようにする,などと言った演出面の改善が挙げられる.

    \end{itemize}

\expandafter\ifx\csname MAIN \endcsname\relax
  \end{document}
\fi


    % 活動で得られたもの
\expandafter\ifx\csname MAIN\endcsname\relax
  \documentclass[a4paper]{jarticle}
  \usepackage[top=2cm, bottom=2cm, left=2cm, right=2cm]{geometry}
  \newcommand{\resp}[1]{\begin{flushright}文責:~#1\end{flushright}~\\}
  \begin{document}
\fi

\section{活動で得られたもの}
  \resp{桐井 優実}

  % ここに文章

\expandafter\ifx\csname MAIN \endcsname\relax
  \end{document}
\fi


    % 展望
\expandafter\ifx\csname MAIN\endcsname\relax
  \documentclass[a4paper]{jarticle}
  \usepackage[top=2cm, bottom=2cm, left=2cm, right=2cm]{geometry}
  \newcommand{\resp}[1]{\begin{flushright}文責:~#1\end{flushright}~\\}
  \begin{document}
\fi

\section{展望}
  \resp{原 佑馬}

\expandafter\ifx\csname MAIN \endcsname\relax
  \end{document}
\fi


\end{document}
